\documentclass[12pt]{exam}
\usepackage{amsmath}
\usepackage[left=1in, right=1in, top=1in, bottom=1in]{geometry}
\usepackage{graphicx}

\newcommand\Cnought{C$_0$}
\newcommand\modulo{\ \texttt{\%}\ }
\newcommand\lshift{\ \texttt{<<}\ }
\newcommand\rshift{\ \texttt{>>}\ }
\newcommand\cgeq{\ \texttt{>=}\ }

\newcommand{\answerbox}[1]{
\begin{framed}
\hspace{5.65in}
\vspace{#1}
\end{framed}}


\pagestyle{head}

\headrule \header{\textbf{15-122 Assignment 6}}{}{\textbf{Page
\thepage\ of \numpages}}

\pointsinmargin \printanswers

\setlength\answerlinelength{2in} \setlength\answerskip{0.3in}

\begin{document}
\addpoints
\begin{center}
\textbf{\large{15-122 : Principles of Imperative Computation
\\ \vspace{0.2in} Summer 1 2012
\\  \vspace{0.2in} Assignment 6
}}

 \vspace{0.2in}
 (\large{Theory Part})

 \vspace{0.2in}

 \large{Due: Monday, June 18, 2012 in class}
\end{center}

\vspace{0.5in}

\hbox to \textwidth{Name:\enspace\hrulefill}


\vspace{0.2in}

\hbox to \textwidth{Andrew ID:\enspace\hrulefill}

\vspace{0.2in}

\hbox to \textwidth{Recitation:\enspace\hrulefill}


\vspace{0.5in}

\noindent The written portion of this week�s homework will give you some practice
working with binary search and AVL trees.
You can either type up your solutions or write them
\textit{neatly} by hand, and you should submit your work in class on the
due date just before lecture begins. Please remember to \textit{staple}
your written homework before submission.
\vspace{0.2in}


\begin{center}
\gradetable[v][questions]
\end{center}


\newpage
\begin{questions}


\pagebreak

\question{\textbf{Treaps}}
Famous Fred Hacker is a huge fan of both binary search trees and heaps.
Given an ordered sequence of $a_1 <
a_2 < ... < a_n$, there are many different binary search trees that
can be produced out of them. Now, suppose we introduce an additional
requirement on the BST: it
has to satisfy the min-heap order property, producing a treap.
More precisely, we assign a priority $p_i$ to each node $a_i$ and
insist that the tree:
\begin{itemize}
\item is a BST with respect to the key values $a_i$
\item has the min-heap order property with respect to the priorities $p_i$
\end{itemize}

\begin{parts}
\part[3] Draw a treap over elements $a < b < c < d$ if we assign
priorities 4,1,2,3 to $a,b,c,d$ respectively?

\begin{solution}
\vspace{2.5in}
\end{solution}

\part[3] What treap do we get if we assign priorities 3,4,1,2 to $a,b,c,d$ respectively?
\begin{solution}
\vspace{2.5in}
\end{solution}

\end{parts}

\newpage


\question{\textbf{Binary Search Trees}}

\begin{parts}

\part[4] Famous Fred Hacker enjoys climbing trees, but he is afraid to do so if 
the tree has height greater than 0. Recall that the height of a binary tree is 
the maximum number of nodes (inclusive) in the path from the root to a leaf.  
For example, a binary tree of just one node has height 1. Write a recursive 
function

\[
    \texttt{int bst\_height(bst B)} \\
\]

that returns the height of a binary search tree. This function itself 
should not be recursive, but you will need to write a helper function that 
\emph{is} recursive. See the BST code developed in lecture for examples of 
wrapper functions and recursive helper functions.  

\begin{solution}
\vspace{6in}

\end{solution}
\newpage

\end{parts}
%
%\newpage
\question{\textbf{AVL Trees}}

\begin{parts}
\part[4] Famous Fred Hacker also likes mudkips.
Draw the AVL tree that results after successively inserting the following keys
in the order shown
\begin{verbatim}
    7  12  10  16  4  2
\end{verbatim}
into an initially empty tree, maintaining and restoring the invariants of a BST 
and the additional balance invariant required for an AVL tree after every 
insert. Your answer should show the tree after each key is successfully 
inserted.


\begin{solution}
  \vspace{7in}
\end{solution}


\part
Famous Fred Hacker has a propensity for coming up with ingenious algorithms, 
but he has a poor sense of balance. One day, he was walking across the street 
and tripped. We want to show that the height of an AVL tree storing $n$ keys is 
$O(\log n)$. For this purpose we count the minimum number of nodes $n$ in an 
AVL tree of height $h$.
\begin{subparts}
\subpart[3] Fill in the table below:
\begin{center}
\begin{tabular}{|c|c|}
 \hline   \hspace{2mm} h  \hspace{2mm} & \hspace{2mm}  n \hspace{2mm} \\[5pt]
 \hline 0 &  1 \\[5pt]
 \hline 1 &  2 \\[5pt]
 \hline 2 &    \\ [5pt] 
 \hline 3 &    \\ [5pt] 
 \hline 4 &    \\ [5pt] 
 \hline
\end{tabular}
\end{center}
\vspace{0.1in}
\subpart[3] Recall that the Fibonacci numbers $F$ are defined by
$$
F(n) = F(n-1) + F(n-2)
$$
$$
F(0) = 0, F(1) = 1
$$

How is the minimum number of nodes $n = T(h)$ in an AVL tree of height $h$ 
related to the Fibonacci numbers?
\begin{solution}
\vspace{0.2in}

$T(h) = $
\vspace{3.2in}
\end{solution}

\end{subparts}

\end{parts}


\end{questions}
\end{document}

